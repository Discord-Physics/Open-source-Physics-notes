\documentclass[10pt,a4paper]{article}
\usepackage[top = 2cm, bottom = 2cm, left = 2cm, right = 2cm]{geometry}

% Mathematical packages
\usepackage{amsmath, amssymb, amsfonts, amsthm}
\usepackage{bm}

% Graphical packages
\usepackage{graphicx}
\graphicspath{{graphics/}}
\usepackage{xcolor}
\definecolor{darkgreen}{rgb}{0.04, 0.30, 0.00}

% Fonts
\usepackage{libertine}
\usepackage[libertine]{newtxmath}
%\usepackage[utf8]{inputenc}

% Endnotes, enumeration
\usepackage{endnotes}
\usepackage{enumitem}

% Parts, sections, subsections ...
\usepackage{titlesec}
\titleformat{\section}{\normalfont\huge\bfseries}{Chapter \thesection}{-5em}{\Huge\newline}

% fbox

% Theorem, definition, lemma, postulate ....
\newcounter{numtheorem}

\newcommand{\theorem}[2]{
    \refstepcounter{numtheorem}
    \color{blue}
    \noindent\fbox{\textbf{1.\thenumtheorem }} \textbf{#1:}
    \color{black}
    #2\newline
}

\newcommand{\definition}[2]{
    \refstepcounter{numtheorem}
    \color{darkgreen}
    \noindent\fbox{\textbf{1.\thenumtheorem }} \textbf{#1:}
    \color{black}
    #2\newline
}

\newcommand{\problem}[2]{
    \refstepcounter{numtheorem}
    \color{red}
    \noindent\fbox{\textbf{1.\thenumtheorem }} \textbf{#1:}
    \color{black}
    #2\newline
}

\renewcommand{\proof}[1]{%
    \noindent\textbf{Proof:}~#1%
    \newline%
    \strut\hfill%
    $\square$%
    \newline%
}%

\renewcommand\d{\mathrm d}
\newcommand{\N}{\mathbb{N}}
\newcommand{\Z}{\mathbb{Z}}
\newcommand{\Q}{\mathbb{Q}}
\newcommand{\R}{\mathbb{R}}
\newcommand{\C}{\mathbb{C}}

\setlength\parindent{0pt}

\begin{document}

\tableofcontents
~\newpage

\subsection{Required knowledge}

\begin{enumerate}
    \item Inner product $(\cdot,\cdot)$, norm $||\cdot||$.
    \item Hilbert space $H$, Banach space.
\end{enumerate}

\newpage
\section{Abstract Fourier series}

\subsection{Complete Orthogonal System}

\definition{Definition}{Let $H$ be a Hilbert space equiped with norm $||\cdot||_H=\sqrt{(\cdot,\cdot)_H}$ and $\{\Phi_\alpha\}_{\alpha\in I}$ be a system of it's elements where $I$ is a set of indicies.
\begin{itemize}
    \item We say that $\{\Phi_\alpha\}_{\alpha\in I}$ is an {\color{darkgreen}\emph{orthogonal system}} if $\forall \alpha\in I:\Phi_\alpha\neq 0$ and $\forall\alpha,\beta,\,\alpha\neq\beta:\left(\Phi_\alpha,\Phi_\beta\right)_H=0$.
    \item We say that $\{\Phi_\alpha\}_{\alpha\in I}$ is an {\color{darkgreen}\emph{orthonormal system}} if it's an orthogonal system and $\forall\alpha:\left(\Phi_\alpha,\Phi_\alpha\right)_H=1$.
    \item We say that $\{\Phi_\alpha\}_{\alpha\in I}$ is an {\color{darkgreen}\emph{complete system}} if $\forall\alpha\in I,\,\forall\Phi\in H:\left(\Phi,\Phi_\alpha\right)_H=0\Leftrightarrow\Phi=0$.
\end{itemize}
}

\definition{Definition}{A subset $A$ of a normed topological space $\left(X,||\cdot||_X\right)$ is called {\color{darkgreen}\emph{dense}} in $X$ if $\forall\epsilon>0,\,\forall x\in X,\,\forall y\in A:||y-x||\leq\epsilon$.
}

\definition{Definition}{We say that $\left(X,||\cdot||_X\right)$ is a {\color{darkgreen}\emph{separable set}} if $\exists A\subset X$, where $A$ countable and dense in $X$.
}

\definition{Definition}{We say that Hilbert or Banach space $H$ is a {\color{darkgreen}\emph{separable}} if $\exists\{\Phi_n\}_{n=1}^\infty,\forall\epsilon>0,\forall\Phi\in H,\exists\Phi_n:||\Phi-\Phi_n||_H<\epsilon$, where $\exists\{\Phi_n\}_{n=1}^\infty$ is a counatble system in $X$.
}

\subsection{Abstract Fourier series}

\theorem{Theorem (About the best approximation)}{Let $H$ be a Hilbert space, $\{\Phi_k\}_{k=1}^\infty\subset H$ be an orthonormal system, $f\in H$, $\{c_k\}_{k=1}^\infty:=\{\left(f,\Phi_k\right)_H\}_{k=1}^\infty$ and $\{a_k\}_{k=1}^\infty\subset\R$. $\forall n\in\N$ let $s_n:=\sum_{k=1}^nc_k\Phi_k$ and $t_n:=\sum_{k=1}^na_k\Phi_k$. Then
\begin{enumerate}
    \item $\forall n\in N: ||f-s_n||_H\leq||f-t_n||_H$,
    \item For given $n\in N: ||f-s_n||_H\leq||f-t_n||_H\Leftrightarrow\forall k\in\{1,\ldots,n\}:a_k=c_k$,
    \item $\forall n\in N: ||f-s_n||_H\leq||f-t_n||_H\Leftrightarrow\forall k\in\N:a_k=c_k$.
\end{enumerate}
}
\proof{
\[
\begin{split}
    ||f-t_n||_H^2&=\left(f-\sum_{k=1}^na_k\Phi_k,f-\sum_{k=1}^na_k\Phi_k\right)_H\\
    &=\left(f,f\right)_H-\left(\sum_{k=1}^na_k\Phi_k,f\right)_H-\left(f,\sum_{k=1}^na_k\Phi_k\right)_H+\sum_{k=1}^n\sum_{l=1}^na_k\bar a_l\left(\Phi_k,\Phi_l\right)_H\\
    &=||f||_H^2-\sum_{k=1}^na_k\bar c_k-\sum_{k=1}^n\bar a_kc_k+\sum_{k=1}^n|a_k|^2\\
    &=||f||_H^2-\sum_{k=1}^n|c_k|^2+\sum_{k=1}^n|c_k|^2-\sum_{k=1}^na_k\bar c_k-\sum_{k=1}^n\bar a_kc_k+\sum_{k=1}^n|a_k|^2\\
    &=||f-s_n||_H^2+\sum_{k=1}^n|c_k-a_k|^2\\
\end{split}
\]
}
\newpage
\definition{Definition}{
Let $H$ be a Hilbert space, $\{\Phi_k\}_{k=1}^\infty\subset H$ be an orthonormal system, $f\in H$ and $\{c_k\}_{k=1}^\infty:=\{\left(f,\Phi_k\right)_H\}_{k=1}^\infty$. Then
\[
        \sum_{k=1}^\infty c_k\Phi_k
\]
is called an {\color{darkgreen}\emph{abstract Fourier series}} with respect to the othonotmal system $\{\Phi_k\}_{k=1}^\infty$. Number $c_k$ is called {\color{darkgreen}\emph{$k$-th Fourier coeficient}}. Abstract Fourier series constructed by this fenition will be denoted as
\[
        f\sim\sum_{k=1}^\infty c_k\Phi_k.
\]
}

\theorem{Theorem (Bessel inequality and Parseval equality)}{
Let $H$ be a Hilbert space, $\{\Phi_k\}_{k=1}^\infty\subset H$ be an orthonormal system, $f\in H$, $f\sim\sum_{k=1}^\infty c_k\Phi_k$ and $s_n:=\sum_{k=1}^n c_k\Phi_k$ for all $n\in\N$. Then
\[
    \sum_{k=1}^\infty|c_k|^2<\infty\text{  and  }\sum_{k=1}^\infty|c_k|^2\leq||f||_H^2\qquad\text{(Bessel inequality),}
\]
and
\[
    \sum_{k=1}^\infty|c_k|^2=||f||_H^2\Leftrightarrow\lim_{n\to\infty}||f-s_n||_H=0\qquad\text{(Parseval equality).}
\]
}
\proof{
Let $n\in\N$. In the proof of theorem about the best approximation let $a_k=c_k$, then
\[
    0\leq||f-s_n||_H^2=||f||_H^2-\sum_{k=1}^\infty|c_k|^2.
\]
Therefore the series $\sum_{k=1}^\infty|c_k|^2$ is bounded by $||f||_H^2$ and by Bolzano-Weierstrass theorem is convergent\footnote{This is not right, it's more like nonegative bounded sequence is convergent while Bolzano-Weierstrass theorem says that every bounded sequence has a convergent subsequence.}. From the equation $||f-s_n||_H^2=||f||_H^2-\sum_{k=1}^\infty|c_k|^2$ we have Parseval equality.
}

\theorem{Riesz–Fischer theorem (reversed Parseval equality)}{
Let $H$ be a Hilbert space, $\{\Phi_k\}_{k=1}^\infty\subset H$ be an orthonormal system and $\{c_k\}_{k=1}^\infty\in\ell^2$. Then
\begin{enumerate}
    \item $\sum_{k=1}^\infty c_k\Phi_k$ converges in $H$ (in other words $\exists f\in H$ such that $f=\lim_{n\to\infty}\sum_{k=1}^n c_k\Phi_k$),
    \item $c_k=\left(f,\Phi_k\right)$,
    \item $\sum_{k=1}^\infty|c_k|^2=||f||_H^2$.
\end{enumerate}
}

\proof{
Basically, we want to show that for $\bm c\in\ell^2,\;\exists f\in H$ such, that $c_k=\left(f,\Phi_k\right)$ and $||f-s_n||^2_H\to 0$ where $s_n=\sum_{k=1}^n c_k\Phi_k$. However $s_n$ is Cauchy sequence\footnote{$s_n$ is Cauchy sequence $\Leftrightarrow$ $\forall\epsilon>0,\;\exists n_0,\;\forall m,n>n_0:|s_m-s_n|<\epsilon$.}, becasue
\[
    ||s_{n+p}-s_n||^2_H=\Big|\Big|\sum_{k=n+1}^{n+p} c_k\Phi_k\Big|\Big|^2_H=\sum_{k=n+1}^{n+p}|c_k|^2<\epsilon
\]
and $H$ is also a Banach space, hence $s_n$ converges to $f\in H$. Now we will show that $c_k=\left(f,\Phi_k\right)$. But
\[
    |c_k-\left(f,\Phi_k\right)|=|\left(s_n,\Phi_k\right)-\left(f,\Phi_k\right)|=|\left(s_n-f,\Phi_k\right)|\leq||s_n-f||_H||\Phi_k||_H=||s_n-f||_H\to 0,
\]
where we used Cauchy-Schwarz inequality and orthonormality of $\{\Phi_k\}_{k=1}^\infty$. Now we have all the requirements for the Parseval equality, hence third statement holds.
}

\newpage
\theorem{Characteristic of complete orthonormal system}{
Let $H$ be a Hilbert space and $\{\Phi_k\}_{k=1}^\infty\subset H$ an orthonormal system. Then following statements are equivalent
\begin{enumerate}
    \item $\{\Phi_k\}_{k=1}^\infty$ is complete system,
    \item $\forall f\in H:\;||f-s_n||_H\to 0$ where $s_n:=\sum_{k=1}^\infty c_k\Phi_k$,
    \item $\forall f\in H:\;\sum_{k=1}^\infty|c_k|^2=||f||_H^2$,
    \item span $\{\Phi_1,\ldots,\Phi_n\}$ for $n\in\N$ is dense in $H$,
\end{enumerate}
where $c_k:=\left(f,\Phi_k\right)$.
}

\proof{
First let's prove $1.\Rightarrow 2.$: We already disscused that $s_n$ is Cauchy sequence in Banach space, hence $||s_n-z||_H\to 0$.. However
\[
    \left(z,\Phi_k\right)=\lim_{n\to\infty}\left(s_n,\Phi_k\right)=c_k=\left(f,\Phi_k\right)\quad\forall k\in\N
\]
therefore $\left(z-f,\Phi_k\right)=0$ and because of system $\{\Phi_k\}_{k=1}^\infty$ is complete we have $f=z$, what means $||s_n-f||_H\to 0$. $2.\Leftrightarrow 3.$ we have from Parselval equality. Now we will show that $3.\Rightarrow 1.$: Consider $f\in H$ such, that $c_k=\left(f,\Phi_k\right)=0,\forall k\in\N$. Hence from Parseval equality we have $||f||_H=0$ and $\{\Phi_k\}_{k=1}^\infty$ is complete system. Statement $2.\Rightarrow 4.$ is true, because $s_n$ is a countable linear combination of elemetns from $\{\Phi_k\}_{k=1}^\infty$ and if $\forall f\in H:\;||f-s_n||_H\to 0$ then $\{\Phi_k\}_{k=1}^\infty$ has to be dense in $H$. Finally $4.\Rightarrow 1.$: Let $z\in H$ be such that $\left(z,\Phi_k\right),\forall k\in\N$ and we want to show that $z=0$. From our assumption there exists $\{t_n\}_{n=1}^\infty,\;t_n\in\text{span}\{\Phi_1,\ldots,\Phi_n\}$ for given $n\in\N$ such that $t_n\to z$ in $H$. Therefore
\[
    ||z||^2_H=\left(z,z\right)=\left(\lim_{n\to\infty}t_n,z\right)=\lim_{n\to\infty}\left(t_n,z\right)=0.
\]
}

\theorem{Corollary}{
    Every Hilbert space in which exists a complete ortonormal system is separable.\\
}
\proof{
    From charakteristic of complete orthonormal systems, statement $1.\Rightarrow 4.$. As a countable, dense subset of $H$ we can take linear combinations of $\{\Phi_k\}_{k=1}^\infty$ with rational coeficients.
}

\theorem{Theroem (On the existence of a complete system in a separable Hilbert space)}{
    In every separable Hilbert space there exists a complete ortonormal system.\\
}
\proof{
    We will distinguish two cases. First, let $H$ be Hilbert's space of finite dimension. Then just take any orthonormal basis. Now let $H$ have an infinite dimension. Due to the assumed separability, there is a large dense subset in it. Let's arrange its elements in the sequence $\{x_j\}_{j=1}^\infty$ and remove all trivial and lineary dependent elements. Complete ortonormal system will be now constructed by using Gram-Schmidt ortogonalization process. Let $\Phi_1:=x_j/||x_j||_H$. Next we will take
    \[
        y_2=x_2-\left(x_2,\Phi_1\right)\Phi_1\quad\text{and}\quad \Phi_2=\frac{y_2}{||y_2||_H}.
    \]
    Rest of elemetns will be constructed by indution.
    \[
        y_{n+1}=x_{n+1}-\sum_{j=1}^n\left(x_{n+1},\Phi_j\right)\Phi_j\quad\text{and}\quad \Phi_{n+1}=\frac{y_{n+1}}{||y_{n+1}||_H}.
    \]
    We can see that $\left(x_1,x_2,\ldots,x_n\right)\subset\text{span}\left(\Phi_1,\Phi_2,\ldots,\Phi_n\right)$, hecne $\text{span}\left(\Phi_1,\Phi_2,\ldots,\Phi_n\right)$ is dense in $H$ and orthonormal. Therefore by statement $4.\Rightarrow 1.$ of theorem "characteristic of complete orthonormal system" $\{\Phi_k\}_{k=1}^\infty$ is also complete.
}

\problem{Solved problem}{
    Consider a Hilert space $\ell^2$ defined as
    \[
        \ell^2:=\Big\{\{x_k\}_{k=1}^\infty\subset\R\;\Big|\sum_{k=1}^\infty x_k<\infty\Big\},
    \]
    with an inner product $\left(\{x_k\}_{k=1}^\infty,\{y_k\}_{k=1}^\infty\right)=\sum_{k=1}^\infty x_ky_k$. For every $n\in\N$ we define $\Phi_n\in\ell^2$ by enumeration as $\left(\Phi_n\right)_k=\delta_{nk}$. Then system $\{\Phi_n\}_{n=1}^\infty$ is a complete ortonormal system in $\ell^2$. Therefore $\ell^2$ is also separable.
}

\definition{Definition}{
    Let $(P_1,\rho_1)$, $(P_2,\rho_2)$ be metric spaces and $I:P_1\to P_2$ a map between them. We say that $I$ is {\color{darkgreen}\emph{isometry}} if $\forall x,y\in P_1:\rho_2\left(I(x),I(y)\right)=\rho_1(x,y)$. Furthemore, we say that spaces $(P_1,\rho_1)$, $(P_2,\rho_2)$ are {\color{darkgreen}\emph{isometric}}, if there exists an isometry mapping $P_1$ onto $P_2$.
}

Isometries of the plane are linear transformations preserving length such as rotation, translation or reflection.\\

\theorem{Corollary}{
    An isometry is always one-to-one by the following argument
    \[
        \forall x,y\in P_1,x\neq y\Rightarrow\rho_1(x,y)\neq 0\Rightarrow\rho_2(I(x),I(y))\neq 0\Rightarrow I(x)\neq I(y).
    \]
    Isometry $I$ is always "one-to-one" and if $(P_1,\rho_1)$, $(P_2,\rho_2)$ are isometric, it is also "onto". Then exists $I^{-1}:P_2\to P_1$ mapping $P_2$ onto $P_1$.
}

\theorem{Theroem (On the relation of separable Hilbert spaces and $\ell^2$)}{
    Every infinitely dimensional separable Hilbert space is isometric to $\ell^2$.\\
}
\proof{
    We know that in the separable Hilbert space $H$ there exists a complete orthonormal system. Let $I$ be a map that assigns to every $f\in H$ a sequence of its Fourier coefficients $\{c_k\}_{k=1}^\infty$ with respect to the given system. Such $I$ is defined on the whole $H$, and is mapping whole $H$ onto $\ell^2$ (if $f\in H$ is finite we have that $\sum_{k=1}^\infty|c_k|^2<\infty$). Therefore, from Parseval's equality
    \[
        ||f||^2_H=\sum_{k=1}^\infty|c_k|^2=||\{c_k\}_{k=1}^\infty||_{\ell^2}=||I(f)||^2_{\ell^2}.
    \]
    Finally, because of linearity of the inner product we have
    \[
        \forall  f,g\in H:||I(f)-I(g)||^2_{\ell^2}=||I(f-g)||^2_{\ell^2}=||f-g||^2_H.
    \]
}

\subsection{Orthogonal projections of Hilbert space}

\definition{Definition}{
    We say that $M$ is a {\color{darkgreen}\emph{closed subspace}} of $H$ if
    \begin{enumerate}
        \item $M\subset H$
        \item $\forall\alpha,\beta\in\C,\;\forall u,v\in M:\alpha u+\beta v\in M$
        \item $\{u_n\}_{n=1}^\infty\subset M$ and $||u_n-u||_H\to 0\,\;\Rightarrow u\in M$
    \end{enumerate}
}

\problem{Solved problem}{
    Let $\Omega\subset\R^N$ be a non-empty open set such, that $\lambda_N(\Omega)<\infty$. Define set
    \[
        M=\Big\{f\in L^2(\Omega)\Big|\int_\Omega f\d x\Big\},
    \]
    then $M$ is a subspace $L^2(\Omega)$. Furthemore if $\{f_n\}_{n=1}^\infty\subset M$ and $f_n\to f$ in $L^2(\Omega)$ we have from Holder inequality
    \[
    \begin{split}
        \Big|\int_\Omega f\d x\Big|&\leq \Big|\int_\Omega(f-f_n)\d x\Big|+\Big|\int_\Omega f_n\d x\Big|=\Big|\int_\Omega(f-f_n)\d x\Big|\leq \int_\Omega|f-f_n|\d x\\
        &\leq ||f-f_n||_{L^2(\Omega)}|1||_{L^2(\Omega)}\to 0.
    \end{split}
    \]
    Hence $f\in M$ and $M$ is closed subspace of $L^2(\Omega)$.
}

\theorem{Theorem (About orthogonal projection)}{
    Let $H$ be a Hilber space and $M$ it's closed subspace.\\ Then $\forall f\in M,\;\exists!f_M\in M:||f-f_M||_H=\inf_{z\in M}||f-z||_H$. We define map $P:H\to M$ such, that $P(f)=f_M$. Then following holds:
    \begin{enumerate}
        \item $P(H)=M$
        \item $P\circ P=P$
        \item for every $z\in M$ holds $z\in P(f)\Leftrightarrow\left(f-z,y\right)_H=0,\forall y\in M$
        \item $\forall f\in H:||f||_H^2=||P(f)||_H^2+||f-P(f)||_H^2$
    \end{enumerate}
}
\proof{

}

\end{document}
