\documentclass[10pt,a4paper]{article}
\usepackage[top = 2cm, bottom = 2cm, left = 2cm, right = 2cm]{geometry}

% Mathematical packages
\usepackage{amsmath, amssymb, amsfonts, amsthm}
\usepackage{bm}

% Graphical packages
\usepackage{graphicx}
\graphicspath{{graphics/}}
\usepackage{xcolor}[dvipsnames]

% Fonts
\usepackage{libertine}
\usepackage[libertine]{newtxmath}
%\usepackage[utf8]{inputenc}

% Endnotes, enumeration
\usepackage{endnotes}
\usepackage{enumitem}

% Parts, sections, subsections ...
\usepackage{titlesec}
\titleformat{\section}{\normalfont\huge\bfseries}{Chapter \thesection}{-5em}{\Huge\newline}

% fbox

% Theorem, definition, lemma, postulate ....
\newcounter{numchapter}

\newcounter{numtheorem}

\newcommand{\theorem}[2]{
    \refstepcounter{numtheorem}
    \color{blue}
    \noindent\fbox{\textbf{\thenumchapter.\thenumtheorem }} \textbf{#1:}
    \color{black}
    #2\bigskip
}

\newcommand{\definition}[2]{
    \refstepcounter{numtheorem}
    \color{green}
    \noindent\fbox{\textbf{\thenumchapter.\thenumtheorem }} \textbf{#1:}
    \color{black}
    #2\bigskip
}

\renewcommand{\proof}[1]{%
    \noindent\textbf{Proof:}~#1%
    \newline%
    \strut\hfill%
    \blacksquare%
    \bigskip%
}%

\renewcommand\d{\mathrm d}
\newcommand{\N}{\mathbb{N}}
\newcommand{\Z}{\mathbb{Z}}
\newcommand{\Q}{\mathbb{Q}}
\newcommand{\R}{\mathbb{R}}
\newcommand{\C}{\mathbb{C}}

\setlength\parindent{0pt}

\title{Classical Mechanics}
\author{Discord Physics \\ organized by R\'{o}bert Jur\v{c}o}
\date{In developement - 2021}


\begin{document}

\maketitle
~\newpage
\tableofcontents
~\newpage

\part{Newtonian mechanics}
\refstepcounter{numchapter}
\section{Kinematics of a Rigid body}

\\~\\

\definition{Definition}{\Emph{Rigid body} .....}

\theorem{Chasles theorem}{General motion of a rigid body has six degrees of freedom, three which represents translational motion of one given point of the rigid body, and three describing rotation around the same point.}

\proof{Here will come the proof.}

\definition{Definition}{\Emph{Basis fixed in space} .....}

\definition{Definition}{\Emph{Co-rotating basis} .....}

\subsection{Angular velocity}

Let $\{\bm e_i\}$ be a basis fixed in space, $\{\bm e'_i(t)\}$ a basis co-rotating with a rigid body. As a co-rotating ortonormal basis is only rotated with respect to a fixed ortonormal basis, we can write $\bm e'_i(t)=A_{ik}(t)\bm e_k$. Consider now an arbitrary, time dependent vector $\bm w=\bm w(t)$ and express it in both bases, $\bm w(t)=w_i(t)\bm e_i=w'_i(t)\bm e'_i(t)$. Time derivative of $\bm w(t)$ with respect to the inertial fixed basis but expressed in the co-rotating bassis is
\begin{equation}
    \frac{\d\bm w}{\d t}=\frac{\d w'_i}{\d t}\bm e'_i+w'_i\frac{\d\bm e'_i}{\d t}=\frac{\d w'_i}{\d t}\bm e'_i+w'_i\frac{\d A_{ik}}{\d t}\bm e_k=\frac{\d w'_i}{\d t}\bm e'_i+w'_i\frac{\d A_{ik}}{\d t}A_{jk}\bm e'_j.
\end{equation}
Therefore, we can formulate a definition:\\

\definition{Definition}{
    Let $\{\bm e_i\}$ be an ortonormal basis\footnote{Both bases are bases of $V$, what is defined as a vector space over $\R$, such that $\dim V=3$.} fixed in space, $\{\bm e'_i(t)\}$ an ortonormal basis co-rotating with a rigid body and $A_{ij}(t)\in C^1\left(\R\right)$ elements of orthogonal matrix\footnote{$A^t=A^{-1}$ and $A_{ik}A_{jk}=\delta_{ij}$} such that $\bm e'_i(t)=A_{ij}(t)\bm e_j$. Then the \emph{tenzor of angular velocity} $\Omega:V\times V\to\R$ is defined as
    \begin{equation}
        \Omega=\frac{\d A}{\d t}A^t,
    \end{equation}
    with elements
    \begin{equation}
        \Omega=\frac{\d A_{ik}}{\d t}A_{jk}.
    \end{equation}
}


\part{Lagrangian and Hamiltonian mechanics}

\part{Fluid mechanics}


\end{document}
